\documentclass{beamer}

\usecolortheme{owl}
\usepackage{booktabs}
\usepackage{makecell}

\begin{document}

\begin{frame}[t]\frametitle{influenced by}
\begin{itemize}
  \item Cinema 4D
  \item TikZ, \LaTeX
  \item CAD
  \item programming (e.g. \texttt{matplotlib}, \texttt{QPainter})
  \item paradigms (cf. programming)
\end{itemize}
\end{frame}

\begin{frame}[t]\frametitle{influenced by Cinema 4D}
\begin{itemize}
  \item intuituve to use
  \item few central managers
  \begin{itemize}
    \item viewport
    \item object tree (objects, generators, tags)
    \item attribute manager
  \end{itemize}
\end{itemize}
\end{frame}

\begin{frame}[t]\frametitle{title}
\begin{itemize}
  \item programming: two assessments
  \begin{itemize}
    \item quality of code (readability, maintainability, \dots)
    \item quality of product (no bugs, speed, usable, \dots)
  \end{itemize}
  \item software engineering priciples (SE-principles):
  \begin{itemize}
    \item avoid code duplication
    \item abstraction
    \item stick to paradigms (procedural, functional, oo, structural, \dots)
    \item apply patterns
  \end{itemize}
\end{itemize}
\end{frame}

\begin{frame}[t]\frametitle{title}
\begin{itemize}
  \item vector graphics: single assessment
  \begin{itemize}
    \item beauty of the result (colors, shapes, content, \dots)
  \end{itemize}
  \item fundamental idea:
  \begin{itemize}
    \item transfer SE-principles to vector graphics
  \end{itemize}
\end{itemize}
\end{frame}

\begin{frame}[t]\frametitle{title}
\begin{itemize}
  \item avoid duplicates
  \item procedural, structural, \textbf{object oriented}
  \item non-destructive
\end{itemize}
\end{frame}

\begin{frame}[t]\frametitle{title}
\begin{itemize}
  \item KISS
  \item user \emph{understands} the application $\Rightarrow$ no unexpected behaviour

  \item very few entities with clear responsibility
  \begin{itemize}
    \item object
    \item viewport
    \item object tree
    \item attribute manager
    \item tag
  \end{itemize}
\end{itemize}
\end{frame}

\begin{frame}[t]\frametitle{object}
\begin{itemize}
  \item has \dots
  \begin{itemize}
    \item \dots a parent and children
    \item \dots attributes
    \item \dots coordinate system (relative to parent, aka. ``local transformation'')
  \end{itemize}
  \item does something
  \begin{itemize}
    \item displays geometry
    \item modifies it's children
    \item acts as group
  \end{itemize}
\end{itemize}
\end{frame}

\begin{frame}[t]\frametitle{objects: empty}
\begin{itemize}
  \item supply a new coordinate system
  \item supply user-attributes
  \item grouping
  \item attributes: name, coordinate system
\end{itemize}
\end{frame}

\begin{frame}[t]\frametitle{objects: rectangle}
\begin{itemize}
  \item shows a rectangle
  \item attributes: like empty + width, height, corer radius, \dots
  \item similar: \emph{ellipse}, \emph{star}, \emph{n-gon}, \dots
  \item can convert to \emph{path}
\end{itemize}
\end{frame}

\begin{frame}[t]\frametitle{objects: path}
\begin{itemize}
  \item shows any path (aka. spline)
  \item attributes: like empty + interpolation, is-closed, \dots
\end{itemize}
\end{frame}

\begin{frame}[t]\frametitle{objects: generators}
\begin{tabular}{lll}
\toprule
object          & adds              & \makecell[l]{attributes\\(plus \emph{empty}-attributes)}  \\
\midrule
\emph{mirror}   & mirrored clone    & -                                                         \\
\emph{cloner}   & many clones       & \makecell[l]{count, shape,\\mode: (linear/circular/path)} \\
\emph{boolean}  & result of boolean & mode: (and, or, xor, not, \dots)                          \\
\emph{instance} & single clone      & ref to source                                             \\
\bottomrule
\end{tabular}
\begin{itemize}
  \item mirrored/cloned object: first children
  \item \emph{boolean}: apply operator to first two (all?) children
  \item \emph{cloner} and \emph{boolean} make hide their children. Only result is visible.
  \item \emph{instance}: no special children
\end{itemize}
\end{frame}

\begin{frame}[t]\frametitle{object tree}
\begin{itemize}
  \item relationship between objects (parent-child) is crucial
  \item encourage use of well-designed object tree dialog
  \item drag-and-drop to set parent/children
  \item select objects
  \item manage tags
\end{itemize}
\end{frame}

\begin{frame}[t]\frametitle{attribute manager}
\begin{itemize}
  \item each object has attributes (see above)
  \item almost every aspect of the scene file is an attribute
  \item display/edit attribute of selected objects
  \item tags have attributes, too
\end{itemize}
\end{frame}

\begin{frame}[t]\frametitle{viewport}
\begin{itemize}
  \item WYSIWYG
  \item select objects
  \item manipulate transformation of object (translate, rotate, scale)
\end{itemize}
\end{frame}

\begin{frame}[t]\frametitle{tags}
\begin{itemize}
  \item attached to any object, each tag knows its owner object.
  \item model features that do not fit into the object tree
  \begin{itemize}
    \item effects: bend, distort, outline, \dots
    \item constrain attribute
    \item style
    \item script
  \end{itemize}
\end{itemize}
\end{frame}

\begin{frame}[t]\frametitle{scripting}
\begin{itemize}
  \item make attribute system availabe to scripting language
  \item python ?
  \item user can define new attributes in any object (user-attribute)
  \item script-tag can access attributes from owner and its children
\end{itemize}
\end{frame}

\begin{frame}[t]\frametitle{templates}
\begin{itemize}
  \item user defines an empty
  \begin{itemize}
    \item with some user-attributes
    \item with script-tag
    \item with some children that ``do'' something
  \end{itemize}
  \item script-tag uses user-attributes to set attributes of the children
  \item \emph{template-object}: like \emph{instance}, but with free top-level attributes
\end{itemize}
\end{frame}

\begin{frame}[t]\frametitle{features}
\begin{itemize}
  \item multi-selection attribute
  \begin{itemize}
    \item usable attribute manager, though multiple objects are selected
    \item display only intersection of attributes
    \item display value only if it is the same
    \item set values like \texttt{+1}, \texttt{*2} smartly
  \end{itemize}
\end{itemize}
\end{frame}

\begin{frame}[t]\frametitle{Open Questions}
\begin{itemize}
  \item start from scratch or extend existing oss?
  \begin{itemize}
    \item Inkscape?
  \end{itemize}
  \item separate material and geometry?
  \begin{itemize}
    \item uncommon in 2D, but common in 3D
  \end{itemize}
  \item file format
  \begin{itemize}
    \item XML, JSON, binary, \dots
  \end{itemize}
\end{itemize}


\end{frame}


\end{document}
